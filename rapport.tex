\documentclass[11pt]{scrreprt}
\usepackage[T1]{fontenc}
\usepackage[utf8]{inputenc}
\usepackage[french]{babel}
\usepackage[scale=0.775]{geometry}
\usepackage{lmodern}
\usepackage[ilines]{scrpage2}
\usepackage[pdftex, bookmarks=true, hidelinks]{hyperref}
\usepackage{graphicx}
\usepackage{tocbibind}
\usepackage{chngcntr}
\usepackage{tabularx}
\usepackage{float}
\usepackage{scrhack}
\usepackage{ulem}
\usepackage{mdframed}

\tolerance=1
\emergencystretch=\maxdimen
\hyphenpenalty=10000
\hbadness=10000

\counterwithout{figure}{chapter}
\counterwithout{table}{chapter}
\pagestyle{scrheadings}

\usepackage[table]{xcolor}
\definecolor{lightgray}{gray}{0.9}
\let\oldtabularx\tabularx
\let\endoldtabularx\endtabularx
\renewenvironment{tabularx}{\rowcolors{2}{white}{lightgray}\oldtabularx}{\endoldtabularx}

\let\oldtabular\tabular
\let\endoldtabular\endtabular
\renewenvironment{tabular}{\rowcolors{2}{white}{lightgray}\oldtabular}{\endoldtabular}

% clear head and foot
\clearscrheadings
\clearscrplain
\clearscrheadfoot

\cefoot[\textsc{Mighty Beards Studio}{\textsc{Mighty Beards Studio}}
\cofoot[\textsc{Mighty Beards Studio}{\textsc{Mighty Beards Studio}}
\lefoot[]{}
\lofoot[]{}
\refoot[\thepage]{\thepage}
\rofoot[\thepage]{\thepage}

\begin{document}

    \renewcommand{\labelitemi}{$\bullet$}
    \renewcommand{\labelitemii}{$\circ$}
    %%%%%% TITLE PAGE %%%%%%%%%%%%%%%%%%%


    %%%%%% TITLE PAGE %%%%%%%%%%%%%%%%%%%
    \begin{titlepage}
        \begin{center}
            %\includegraphics[width=5cm]{images/dia/logo.pdf}
            ~\\[1.5cm]

            %\textsc{\LARGE Mighty Beards Studio}\\[1.5cm]

            \textsc{\Large Projet d'AAE}\\[0.5cm]

            % Title
            \rule{\textwidth}{1pt} \\[0.4cm]
            { \bfseries \Huge{Blitz}\\ \Large{Clone du jeu Wazabi}\\[0.4cm]}

            \rule{\textwidth}{1pt} \\[1.5cm]

            % Authors
            \textsc{Javier Lethé - Gaëtan Navez - Matteo Taroli - Przemyslaw Gasinski}

            \vfill

            {\large \today}
            \vfill
        \end{center}
    \end{titlepage}
    \pagenumbering{gobble}
    \tableofcontents
    \pagebreak
    \pagenumbering{arabic}

    \setlength{\parskip}{3mm}
    %	#	#	#	#	#	#	#	#	#	#	#	#	#	#	#	#	# Document

    \chapter{Analyse}
    \section{Scénario nominal}

    L'utilisateur étant déjà connecté, il rejoint une partie selon le scénario suivant :

    \begin{table}[H]
        \begin{tabularx}{\textwidth}{X|X}
            1. Le joueur clique sur le bouton \og Rejoindre une partie \fg{} & \\
            & 2. Le système vérifie si une partie est déjà lancée \\
            3.a [Une partie est en cours de lancement] & \\
            & a.1 Le système envoie le joueur vers le lobby.\\
            & a.2 Une fois que le nombre minimal de joueur est atteint, le jeu est lancé\\
            3.b [Aucune partie n'est en cours] & \\
            & b.1 Le système informe le joueur et le renvoie vers la page de création de jeu. \\
            b.2 Le joueur crée une partie & \\
            & b.3 Le système envoie le joueur vers le lobby. \\
            & b.4 Une fois que le nombre minimal de joueur est atteint, le jeu est lancé\\
            3.c [Une partie est déjà en cours de jeu] & \\
            & c.1 Le système ne permettant de lancer qu'une partie à la fois, il informe le joueur et le renvoie vers la page d'accueil.\\
        \end{tabularx}
    \end{table}

    \section{Tests Fonctionnels}
    \section{Connexion}
    \begin{table}[H]
        \begin{tabularx}{\textwidth}{X|X}
            1. Cliquer sur le bouton \og Connexion / Inscription\fg{} situé en haut à droite. & \\
            & 2. Un popup apparait proposant de s'inscrire ou de se connecter. \\
            3. Dans la partie de droite, entrer \og test\fg{} comme nom d'utilisateur et \og test \fg{} comme mot de passe. & \\
            & 4. Le système vérifie que l'utilisateur existe et que le mot de passe est correct et renvoie le joueur sur la page d'accueil. \\
            5. Cliquer sur le bouton \og Déconnexion\fg{}. & \\
            6.Cliquer à nouveau sur le bouton \og Connexion / Inscription\fg{}. & \\
            & 7. Le popup réapparait \\
            8. Dans la partie de droite, entrer \og test\fg{} comme nom d'utilisateur et \og mauvaisMdp \fg{} comme mot de passe. & \\
            & 9. Le système vérifie que l'utilisateur existe et que le mot de passe est correct et affiche l'erreur \og Nom d'utilisateur ou mot de passe incorrect\fg{} et reste sur le popup.\\
        \end{tabularx}
    \end{table}
    \section{Création de partie}
    \begin{table}[H]
        \begin{tabularx}{\textwidth}{X|X}
            1. En partant de la page d'accueil, cliquer sur le bouton \og Créer une partie\fg{}. & \\
            & 2. Un popup apparait permettant d'entrer le nom de la partie.\\
            3. Entrer \og test\fg{} en tant que nom de partie. & \\
            4. Cliquer sur le bouton \og OK\fg{}. & \\
            & 5.Le système renvoie vers le lobby de la partie. \\
            6.Attendre d'autres joueurs et joueur. & \\
        \end{tabularx}
    \end{table}

    \section{Jeu}
    \begin{table}[H]
        \begin{tabularx}{\textwidth}{X|X}
            1. Attendre que ce soit 
        \end{tabularx}
    \end{table}


\end{document}
